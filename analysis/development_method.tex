\section{Development Method} % (fold)
\label{sec:development}
\meta{Describes BDD and why I choose it}

%intro
For a software project to be successful it is important to choose the right
development method. Behaviour-Driven Development (BDD) was chosen as the
development method for this project due to it being an agile process and
utilizing test-first development. The following describes BDD and how it suits
this project.

\subsection{Behaviour-Driven Development}
%intro


%cycle, acceptance -> spec -> red/green
An iteration in BDD focus around an acceptance test covering one feature i.e.\
logging functionality. At the beginning of the iteration an acceptance test is
written and executing it should reveal that the functionality is not yet
implemented according to it's requirements. Cucumber is a framework for writing
acceptance tests in a natural language as a set of steps in one or more scenarios.
An example Cucumber acceptance test is included in
Listing~\ref{cucumber_example}.

\bigskip
\begin{lstlisting}[label=cucumber_example,caption=Cucumber acceptance test example.]
Feature: Logging
  As a webdeveloper
  I want to have logging of HTTP requests
  In order to debug errors or attacks on the webserver

  Scenario: Logging of HTTP requests
    Given the webserver is running
    When a user visits http://localhost/posts
    Then the log should contain "GET /posts"
    And the log should contain "SENT /posts SUCCESS"
\end{lstlisting}

The acceptance test covers the logging functionality of a webserver and tests it
by starting the webserver, making a request and then checking whether the log
contains the correct entry.  The first line names the feature covered by the
test. Lines 2 through 4 states the user of the feature, what it should do and
lastly why and what business value it creates. Line 6 states a specific
scenario, and often a feature has multiple scenarios and needs to test them
separately. Scenarios contains separate steps which defines either a
prerequisite (Given), an action (When) or a result (Then)\footnote{"And"
continues the previous step type}. The semantic of the steps are defined in a
set of step definitions written in plain Ruby.
Listing~\ref{step_definition_example} is a step definition of the step used on
line 9 in Listing~\ref{cucumber_example}.

\begin{lstlisting}[label=step_definition_example,caption=Cucumber step definition.]
Then /^the log should contain "([^"]*)"$/ do |entry|
  File.open("log.txt", r) do |line|
    line == entry
  end 
end
\end{lstlisting}

The step definition opens the log file and checks whether it contains given
entry. Steps are matched with regular expressions and \texttt{"([^"]*)"}
matches the string "GET /posts" on line 9 in Listing~\ref{cucumber_example}.
Step definitions returns true if it passes and false if it didn't, and in the
case of a failure displays a detailed debug message.

