In order to have a clear goal of what the software produced in this project
should do, a requirements specification is essential. The following reasons
for the seven requirements for webserver produced in this project.

The webserver is named Yarn, and the latter of the report will refer to it by
this name.

As covered in Section~\ref{webservers}, the main purpose of a webserver is to
listen for HTTP requests and then return a response
according to the request. Therefore, Yarn should be able to parse HTTP
requests in order to know how to respond. The parser should parse HTTP
requests according to the RFC 2616 specification.

In order to analyse and debug Yarn, it should be able to log vital
messages to the console. This feature would greatly improve troubleshooting
and serve as a backlog of events in the case of an error or attack.

To increase the performance and throughput of Yarn it should be able
to handle
multiple requests at a time in a non-blocking manner. This would keep a slow
request from blocking other requests from being processed.

Yarn should be able to serve static files like HTML, cascading
stylesheets, JavaScript and images in order to make webpages work as expected.
If a folder is requested, then Yarn should return a HTML
formatted directory listing with links to the files in the given folder. 

If a resources doesn't exist, or an error occurs during processing, the
webserver should return an error page with a short message on what went wrong.

Yarn should be able to serve dynamic content by executing Ruby
scripts and returning the output. This enables client interaction instead of
simply being served static content.

In order to use Yarn with one of the many Ruby web-frameworks, it
should implement the Rack interface. This would instantly enable developers to
use Yarn with their Ruby web-applications.

\section{Requirements Specification}
\label{req_spec}
The list below summarises the requirements for Yarn.

\begin{itemize}
  \item Parse HTTP/1.1 requests in accordance with RFC 2616.
  \item Logging of requests and responses.
  \item Handle concurrent requests in a non-blocking manner.
  \item Serve static files and directory listings.
  \item Return an error page if an error occurs.
  \item Serve dynamic content by executing Ruby scripts and returning the
  output.
  \item Implement the Rack interface.
\end{itemize}

As this project was developed using BDD, the requirements specification was
translated into a set of acceptance tests to get concrete feedback on
completed requirements. These acceptance tests are included on the CD in
\texttt{Yarn/features}.

