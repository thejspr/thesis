% \meta{Should describe how and why I have decided upon the set of requirements.
%   Given my choice of development method, there is not much design upfront,
%         hence this section will only contain some initial thoughts on the
%           structure of the software.}

In order to have a clear goal of what the software produced in this project
should do, a set of requirements are essential. 

As covered in Section~\ref{webservers}, the main purpose of a webserver is to
listen to HTTP requests, locate, and then return the resource or a response
according to the request. Hence, the webserver should be able to parse HTTP
requests in order to know how to respond. The parser should parse HTTP
requests according to the RFC 2616 specification.

In order to analyse and debug the webserver it should be able to log vital
messages to the console. This feature would greatly improve troubleshooting
and serve as a backlog of events in case of an error or an attack.

To increase the performance and throughput of the webserver it should handle
multiple requests at a time in a non-blocking manner. This would solve a slow
request blocking other requests from being processed.

The webserver should be able to serve static files like HTML, cascading
stylesheets, JavaScript and images in order to make webpages work as expected.
If no specific file is requested, then the webserver should return a HTML
formatted directory listing with links to the files in the given folder.

If a resources doesn't exist, or an error occurs during processing, the
webserver should return an error page with a short message on what went wrong.

The webserver should be able to webserver dynamic content by executing Ruby
scripts and returning their output. This enables client interaction instead of
simply being served static content.

In order to use the webserver with one of the many Ruby web-frameworks, it
should implement the Rack interface. This would instantly enable developers to
use the webserver with their web-applications.

\section{Requirements Specification}
\label{req_spec}
The list below summarises the requirements for the webserver developed in this
project.

\begin{itemize}
  \item Parse HTTP/1.1 requests in accordance with RFC 2616.
  \item Logging of requests and responses.
  \item Handle concurrent requests in a non-blocking manner.
  \item Serve static files and directory listings.
  \item Return an error page if an error occurs.
  \item Serve dynamic content by executing Ruby scripts and returning the
  output.
  \item Implement the Rack interface.
\end{itemize}

As this project was developed using BDD, the requirements specification was
translated into a set of acceptance tests to get concrete feedback on
completed requirements.

The webserver is named Yarn, and the later of the report will refer to it by
this name.
