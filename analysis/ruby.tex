\section{Ruby} % (fold)
\label{sec:ruby}
\metatext{Ruby in general. Concurrent programming in ruby. Describe different
  implementations.}

%ruby in general
Ruby is an dynamic, reflective object-oriented general-purpose programming
language. It is dynamic in the sense that it is interpreted at runtime, and
reflective in the sense that it can modify and inspect program behavior during
runtime.  Ruby has a dynamic type system where the type of the object is
determined by what the object can do in terms of which methods are available
for the object. This type system is referred to as duck typing, i.e.\ if it
walks like a duck and talks like a duck, then the interpreter will treat it
like a duck.

The latest version of Ruby, version 1.9, holds several improvements over the
older but still maintained version 1.8. Besides small syntactic changes and
including a new VM, Ruby 1.9 introduced the concept of fibers.  The following
section covers the options available for concurrent programming in Ruby.

\subsection{Concurrency in Ruby}
Ruby comes with several constructs for concurrent programming; fibers, threads
and processes. The following describes the advantages and possible disadvantages
of each of these concepts.

\subsubsection{Fibers}

\subsubsection{Threads}

\subsubsection{Processes}


\subsection{Ruby Implementations}
Ruby code is run in a virtual machine (VM) of which there exists several
implementations. The official Ruby VM, for Ruby 1.9, is called YARV (Yet 
Another Ruby VM). As there is no specification for Ruby, YARV serves as
the reference implementation for the Ruby language.  There is however a
project called RubySpec which aims to write an executable specification in the
form of unit-tests which, is executed against the Ruby VM implementation to
verify that it executes Ruby correctly. The following describes three widely
used Ruby implementations, namely YARV, Rubinius and JRuby, and how they
facilitate concurrent programming.

\subsubsection{Yet Another Ruby VM}
YARV is developed an maintained as the official Ruby implementation and replaced
the old VM MRI (Matz\footnote{Yukihiro Matsumoto, a.k.a.\ Matz is the creator of
    the Ruby programming language} Ruby Interpreter) as of Ruby version 1.9.

\subsubsection{Rubinius}


\subsubsection{JRuby}
JRuby is 


% section ruby (end)
