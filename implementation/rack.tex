\section{Rack Applications}
\label{rack}
% \meta{describe how the Rack interface was implemented and what it required}
Implementing support for Rack applications consisted of the following two parts: 

\begin{enumerate}
  \item Loading the Rack application from a file.
  \item Communicating requests and responses between the server and the application.
\end{enumerate}


Rack applications are specified in a Ruby file which needs to be evaluated to
load the application. An example application which returns
a response with HTTP status 200, a Content-Type header, and a body with the
test "Rack works", is included in Listing~\ref{rackex}.

\newpage
\begin{lstlisting}[label=rackex,caption=Sample Rack application. (test\_objects/config.ru:4)]
run Proc.new { |env|
  [200, {'Content-Type' => "text/plain"}, ["Rack works"]]
}
\end{lstlisting}

When a Rack application file is supplied when invoking Yarn, the contents of
the file is read, passed into the \texttt{Rack::Builder} constructor which is
then evaluated. Listing~\ref{loadrack} contains the lines from the
\texttt{Server} class where the Rack application is loaded.

\bigskip
\begin{lstlisting}[label=loadrack,caption=Loading a Rack application.
(Yarn/lib/yarn/server.rb:46)]
config_file = File.read(app_path)
rack_application = eval("Rack::Builder.new { #{config_file} }.to_app", TOPLEVEL_BINDING, app_path)
\end{lstlisting}

Once loaded, a Rack application is invoked with a \texttt{call} method which
returns an array containing the HTTP status code, HTTP headers and message
body. The \texttt{call} method takes one parameter which is a
\texttt{Hash} of values the Rack application needs such as the path, hostname,
port, request body etc. \cite{rackspec}. 

