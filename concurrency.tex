\section{Multiprocessor Programming} % (fold)
\label{sec:multiprocessor}
Concurrency refers to code which is executed simultaneusly on one or more processors or cores in a single processor. Concurrent code executed on a single-core processor will look like it's executed at the same time, but in reality the processor executes increments of each piece of code by switching back and forth between them. The way a processor switches between concurrent tasks is referred to as context-switching, and can easily bog down a system as the context (program state and variables) needs to be loaded for every context-switch. 
Concurrent code executed on a multi-core processor has the potential to execute code in parallel, e.g.\ at the same instance in time.

In most programming languages code is executed one statement at a time in a sequential maner. Most programming languages does however include constructs to allow for concurrent programming. 
One of the most commen constructs is \textit{threads}. A thread is an abstraction for a piece of code which is executed as a subroutine of the running program. This allows for executing several threads concurrently or in parallel on multiple processors. 
In Section~\ref{sec:ruby}, the constructs available in Ruby are covered in more detail.

% section multiprocessor (end)
