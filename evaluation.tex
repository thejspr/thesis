This chapters looks critically at the final product and reflects upon the
project process using BDD.

\section{Product Evaluation}
The final version of Yarn (0.1.1) includes all features from the requirement
specification, and all acceptance tests and specifications passes. 

Yarn becomes unstable at very high request volumes. However, when the requests to worker
process ratio becomes too high, some TCP connections are dropped and no
content is delivered to the client.  It was not possible to fix this bug as no
more time was left for development.

The performance of Yarn is quite satisfying for CPU intensive applications,
but for static content it performs subpar to other Ruby webservers. The
performance bottleneck is the HTTP parser, and a switch to using a C extension
parser would be feasible.

\section{Process Reflection}
Developing software tests first proved to be difficult at times when it was
hard to figure out how a piece of code would even work. Having done mostly
non-test-first development prior to this project, it required a continuous
conscious effort to stick to writing tests before implementing features. Late
in the development phase, it became a more natural effort to drive the
implementation through testing, and the advantages of the work-flow were proved
a great asset. Having the tests provided the courage which was sometimes
needed prior to a large rewrite or refactoring of a feature, as it was 
easy to see whether the changes had broken any functionality. Furthermore, by writing
the tests first, it was always clear what the next step was as running the
test would reveal exactly what was missing in the implementation.

Given there was no design phase prior to starting the development, the design
decisions were made continuously throughout the development. This resulted in
some decisions made early on which later proved to be less fortunate. But due
to the agile nature of BDD, and the confidence achieved by having a big test
suite, drastic design changes late in the process were possible. Had the
system been thoroughly designed upfront, such drastic changes might have been
fatal to the success of the project.
