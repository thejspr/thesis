\section{Ruby} % (fold)
\label{sec:ruby}
\metatext{Ruby in general. Concurrent programming in ruby. Describe different
  implementations.}

%ruby in general
Ruby is an dynamic, reflective object-oriented general-purpose programming
language. It is dynamic in the sense that it is interpreted at runtime, and
reflective in the sense that it can modify and inspect program behavior during
runtime.  Ruby has a dynamic type system where the type of the object is
determined by what the object can do in terms of which methods are available
for the object. This type system is referred to as duck typing, i.e.\ if it
walks like a duck and talks like a duck, then the interpreter will treat it
like a duck.

The latest version of Ruby, version 1.9, holds several improvements over the
older but still maintained version 1.8. Besides small syntactic changes and
including a new VM, Ruby 1.9 introduced the concept of fibers. Ruby 1.9 also
includes RubyGems which enables easy packaging, installation and distribution of
Ruby software. RubyGems, mostly referred to as \textit{gems}, can be used to
modify or extend functionality within a Ruby application or to split out
reusable code for others to benefit from. There are currently over 27.000 gems
available and this greatly reduces the need to "reinvent the wheel", hence
improving productivity. 

Ruby was chosen for this project because it is dynamic, expressive, productive
and terse, all adding to the speed and joyfulness of developing software.
Furthermore, Ruby has several constructs available for concurrent programming
which are covered below.

\subsection{Concurrency in Ruby}
Ruby comes with several constructs for concurrent programming; fibers, threads
and processes. The following describes the advantages and possible disadvantages
of each of these concepts.

\subsubsection{Threads}


\subsubsection{Fibers}
A fiber in Ruby is a coroutine mechanism which enables pausing and resuming
code blocks to achieve cooperative concurrency. A code block is a
piece of ruby code encapsulated in an object, and can be passed into
methods, saved in variables, and executed on demand. Fibers resembles threads in
that they can be controlled from outside, but whereas threads are managed by a
scheduler, fibers must be scheduled by the programmer \cite{rubyfiber}.

The advantages of fibers over threads are that they have a significantly lower
memory footprint and doesn't introduce the complexity of a scheduler as
threads do.


\subsubsection{Processes}


\subsection{Ruby Implementations}
Ruby code is run in a virtual machine (VM) of which there exists several
implementations. The official Ruby VM, for Ruby 1.9, is called YARV (Yet 
Another Ruby VM). As there is no specification for Ruby, YARV serves as
the reference implementation for the Ruby language.  There is however a
project called RubySpec which aims to write an executable specification in the
form of unit-tests which, is executed against the Ruby VM implementation to
verify that it executes Ruby correctly. The following describes three widely
used Ruby implementations, namely YARV, Rubinius and JRuby, and how they
differ concurrent programming.
\fixme{sources}

\subsubsection{Yet Another Ruby VM}
YARV is developed an maintained as the official Ruby implementation and replaced
the old VM MRI (Matz\footnote{Yukihiro Matsumoto, a.k.a.\ Matz is the creator of 
the Ruby programming language} Ruby Interpreter) as of Ruby version 1.9. Given
YARV is the one driving the Ruby language implementation, it always has the
latest languages features and updates, whereas other Ruby implementations tend
to be a bit behind. The latests stable version of YARV is 1.9.2, with 1.9.3
being just around the corner.
\fixme{sources}

\subsubsection{Rubinius}
\fixme{Decide whether to include Rubinius in tests, otherwise don't describe
  it}

\subsubsection{JRuby}
JRuby is a Ruby implementation that runs on any Java Virtual Machine (JVM)
and therefore enables using Java libraries from within Ruby. Java is a very
mature and large ecosystem with many useful libraries, language constructs and
concurrency models.
\fixme{How does JRuby exactly work?}

JRuby does not include a GIL as YARV does and therefore allows true
parallelism as opposed to only having one thread/fiber/process run at a time.
JRuby, as of version 1.6.4, supports both Ruby 1.8 and 1.9.
\fixme{sources}


% section ruby (end)
