% \meta{A brief introduction to the project, with a clear definition 
% of the problem to be addressed, and guide to rest of the report.}
%
% What the thesis will say
% make it clear in the introduction that you want to use multi-core/thread to speed up the web server.
As websites must be able to service multiple visitors at a time, the
webserver serving it must employ some form of concurrency. Various concurrency
models exist, and each comes with a set of advantages and disadvantages, all
depending on the context of the webserver and the content being served.

This project concerns developing a concurrent webserver in the Ruby
programming language, using Behaviour-Driven Development.

\section{Readers Guide}
Background information required for the remainder of the report is covered in
Chapter~\ref{bg}. Chapter~\ref{an} covers the requirements for the software
developed. Chapter~\ref{im} regards implementation specific aspects of the
software and Chapter~\ref{te} describes the testing done during the
development. In Chapter~\ref{de}, the end product is demonstrated,
evaluated through a series of benchmarks and compared to other Ruby
webservers. In Chapter~\ref{re} the project process is reflected upon, and
Chapter~\ref{co} concludes the project and suggests future improvements to the
software.

For source code listings, the caption will show which file and on what line
the code is taken from. The source code is included on a CD, and also
available at \url{https://github.com/thejspr/yarn}.

\section{Acknowledgements}
I would like to thank Kai Xu for invaluable supervision and feedback
throughout this project.
